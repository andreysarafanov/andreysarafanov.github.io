\documentclass{beamer}

\mode<presentation>
{
  \usetheme{Antibes}
  \usecolortheme{beaver}
}

\newcommand{\bs}{\textbackslash}


\usepackage[english, russian]{babel}
\usepackage[utf8]{inputenc}
\usepackage{times}
\usepackage[T2A]{fontenc}
\usepackage{pscyr}
\usepackage{amsthm,amsfonts,amsmath,amssymb,amscd}
\usepackage{courier}
\usepackage{verbatim}



\title[Презентации в PDF] {Создание презентаций при помощи Beamer. 
Использование графики в \LaTeX.}

\author {Андрей Сарафанов, группа 524}

\institute {\small{
  Факультет вычислительной математики и кибернетики\\
  МГУ им. Ломоносова
}}

\date {\footnotesize{Москва, 2015}}
\begin{document}

\begin{frame}
  \titlepage
\end{frame}

\begin{frame}{План доклада}
  \tableofcontents
\end{frame}

\section{Beamer}

\subsection{Основная информация про Beamer}

\begin{frame}{Что такое Beamer?}
Beamer - один из классов Latex, позволяющих создавать презентации.
Beamer
  \begin{itemize}
  \item
    поставляется со множеством шаблонов, которые можно детально настраивать,
  \item
    предоставляет множество дополнительных команд, дающих контроль над содержимым слайдов.
  \end{itemize}
\end{frame}

\begin{frame}{
    Пример возможностей Beamer
  }{
    Можно управлять расположением элементов на экране
  }
  \begin{columns}
    \column{.40\textwidth}
      \begin{block}{первая колонка}
        \begin{itemize}
          \item первый левый
          \item второй левый
        \end{itemize}
      \end{block}
    \column{.40\textwidth}
      \begin{block}{вторая колонка}
        \begin{itemize}
          \item первый правый
          \item второй правый
        \end{itemize}
      \end{block}
    \pause
    \column{.2\textwidth}
      \begin{block}{третья колонка}
        Содержимое самой правой колонки.
      \end{block}
  \end{columns}
\end{frame}


\subsection{Детальный разбор Beamer}

\begin{frame}{Установка Beamer}
  Скорее всего, если LaTeX в системе настроен, то Beamer уже будет установлен.\\
  \textbf{Зависимости}: pgf.sty 1.00+, xcolor.sty 2.00+, [pdflatex 0.14+]
  \textbf{Установка}: 
    \begin{itemize}
      \item Mac\TeX\space и \TeX\space Live: утилитой \texttt{tkmgr}, в свежих версиях будет изначально
      \item Mik\TeX\space и pro\TeX t: средствами этих IDE
      \item Debian, Ubuntu: \texttt{apt-get install latex-beamer}
      \item Fedora: \texttt{yum install texlive-texmf-latex}
      \item В любой системе можно установить вручную, подробнее в документации/
    \end{itemize}
\end{frame}

\begin{frame}[fragile]{Выбор макета и цветовой схемы презентации}
  \begin{columns}
    \column{.5\textwidth}
        К Beamer в комплекте идёт ряд макетов и цветовых схем.\\
        При желании, настройки цвета и макет можно изменить как глобально,
        так и для конкретного слайда.
    \column{.5\textwidth}
      \begin{verbatim}
        \documentclass{beamer}

        \mode<presentation>
        {
          \usetheme{Warsaw}
          \usecolortheme{fly}
        }  
      \end{verbatim}
  \end{columns}
\end{frame}

\begin{frame}[fragile]{Данные для титульного листа}
  \begin{columns}
    \column{.3\textwidth}
      Ещё перед первым слайдом добавляется информация о презентации. Она участвует в генерации шаблона и титульного слайда (\bs titlepage).
    \column{.7\textwidth}
      \begin{verbatim}
        \title[Презентации в PDF] {
          Создание презентаций 
          при помощи Beamer. 
          Использование графики в \LaTeX.
        }
           \author {Андрей Сарафанов, группа 524}
        \institute {
          Факультет вычислительной математики 
          и кибернетики\\
          МГУ им. Ломоносова
        }
        \date {\footnotesize{Москва, 2015}}
      \end{verbatim}
  \end{columns}
\end{frame}

\begin{frame}[fragile]{Создание слайдов}
  \begin{columns}
    \column{.3\textwidth}
      Каждый слайд обрамляется в
      \texttt{
        \bs begin\{frame\}[arguments]\\\{title\}\{subtitle\}
        \\[1\baselineskip]
        \bs end\{frame\}
        }
    \column{.7\textwidth}
      \begin{verbatim}
        \title[Презентации в PDF] {
          Создание презентаций 
          при помощи Beamer. 
          Использование графики в \LaTeX.
        }
           \author {Андрей Сарафанов, группа 524}
        \institute {
              Факультет вычислительной математики 
          и кибернетики\\
          МГУ им. Ломоносова
        }
        \date{Москва, 2015}
      \end{verbatim}
  \end{columns}
\end{frame}

\begin{frame}[fragile]{Управление видимостью объектов}
  \begin{columns}
    \column{.3\textwidth}
      Команда \texttt{\bs pause} может быть использована, чтобы разбить слайд на набор кадров, которые будут выполняться по очереди.\\
      \textit{Скоро появится список}
      \pause
      \begin{itemize}
        \item1 Первый элемент
        \pause
        \item2 Второй элемент
      \end{itemize}
      \pause
      \textit{Список закончился}
    \onslide
    \column{.7\textwidth}
      \begin{verbatim}
      \textit{Скоро появится список}
      \pause
      \begin{itemize}
        \item1 Первый элемент
        \pause
        \item2 Второй элемент
      \end{itemize}
      \pause
      \textit{Список закончился}
      \end{verbatim}
  \end{columns}
\end{frame}

\begin{frame}[fragile]{Управление видимостью объектов}
  Ряд команд позволяет указывать, на каких кадрах они выполняются.\\
  \texttt{<1-3,5,6,8-> обозначает все кадры кроме 4 и 7}
  \only<1>{
    \begin{itemize}
      \item \texttt{\bs textbf, \bs textit, \bs textsl, \bs textrm, \bs textsf, \bs color, \bs alert, \bs structure}: эти команды игнорируются на кадрах, не указанных в спецификации,
      \item \texttt{\bs only}: показывает текст только на указанных кадрах, на других он игнорируется, не занимает вертикального пространства,
      \item \texttt{\bs uncover}: на кадрах, не указанных в спецификации, текст становится невидимым или полупрозрачным в зависимости от настроек,
      \item \texttt{\bs visible}: на кадрах, не указанных в спецификации, текст становится невидимым,
      \item \texttt{\bs invisible}: противоположность \texttt{\bs visible}
    \end{itemize}
  }
  \only<2->{
    \begin{itemize}
      \item \texttt{\bs alt}: принимает два параметра, на указанных слайдах показывает первый, на остальных - второй,
      \item \texttt{\bs label}: можно указывать только один кадр, на котором элемент существует (по умолчанию --- первый кадр), используется в основном для навигации к нужному слайду,
      \item \texttt{\bs item}: почти аналогично оборачиванию элемента в \texttt{\bs uncover} с той же спецификацией,
      \item \texttt{\bs temporal}: принимает 3 параметра. Для кадров, не входящих в спецификацию, но более ранних, чем какой-либо кадр из неё, показывается первый, для кадров из спецификации - второй, иначе третий.
    \end{itemize}
  }
\end{frame}
\begin{frame}[fragile]{Управление видимостью объектов}
  Помимо команд, спецификации видимости могут использоваться со многими окружениями.
  \begin{itemize}
    \item \texttt{theorem, proof, example,..}: для большинства окружений указание спецификации аналогично оборачиванию в \texttt{\bs uncover}
    \item \texttt{uncoverenv, visibleenv, invisibleenv}: аналогично соответственно \texttt{\bs uncover}, \texttt{\bs visible}, \texttt{\bs invisible}
    \item \texttt{onlyenv}: аналогично \texttt{\bs only}, но может показывать ошибки, которые \texttt{\bs only} бы "проглотила" (см. док.)
    \item \texttt{altenv}: принимает 4 параметра: \texttt{\{begin text\}\{end text\}\{alternate begin text\}\{alternate end text\}}
  \end{itemize}
  При помощи \texttt{\bs only} можно динамически менять содержимое слайда, но это может привести к скачкам по вертикали. Избежать этого позволяют окружения \texttt{overlayarea, overprint} (см.док.)
\end{frame}

\begin{frame}{
    Примеры управления видимостью элементов.
  }{
    Управление видимостью при помощи \texttt{\bs pause}
  }
  \begin{columns}
    \column{.4\textwidth}
      Последовательно показываем элементы списка при помощи 
      \begin{itemize}
        \item пункт первый
        \pause
        \item пункт второй
        \pause
      \end{itemize}
      {Вывод на основе пунктов}
      \onslide
    \column{.6\textwidth}
      \begin{verbatim}
      \begin{itemize}
        \item пункт первый
        \pause
        \item пункт второй
        \pause
      \end{itemize}
      Вывод на основе пунктов
      \end{verbatim}
  \end{columns}
\end{frame}


\begin{frame}{
    Примеры управления видимостью элементов.
  }{
    Управление видимостью при помощи спецификаций видимости.
  }
  \begin{columns}
    \column{.4\textwidth}
      \begin{itemize}
        \item пункт виден всегда
        \only<3->\item пункт был спрятан до 3 кадра
        \item<2-> пункт второй, спрятан до второго кадра
      \end{itemize}
      \begin{altenv}<1-2>{[}}{]}{(}{)}
        Первые два кадра текст в квадратных скобках, потом в круглых
      \end{altenv}
      \onslide
    \column{.6\textwidth}
      \begin{verbatim}
      \begin{itemize}
        \item пункт виден всегда
        \only<3->\item пункт был спрятан до 3 кадра
        \item<2-> пункт второй, спрятан до второго кадра
      \end{itemize}
      \begin{altenv}<1-2>{[}}{]}{(}{)}
        Первые два кадра текст в квадратных 
        скобках, потом в круглых
      \end{altenv}
      \end{verbatim}
  \end{columns}
\end{frame}

\begin{frame}[fragile]{Разбиение на колонки}
  \begin{columns}
    \column{.4\textwidth}
      Содержимое слайда можно разбивать на колонки с заданной шириной.
      \begin{columns}
        \column{.4\textwidth}
        И даже можно
        \column{.6\textwidth}
        делать это рекурсивно
      \end{columns}
    \column{.6\textwidth}
      \begin{verbatim}
        \begin{columns}
          \column{.4\textwidth}
            Слева
            \begin{columns}
              \column{.4\textwidth}
              слева-слева
              \column{.6\textwidth}
              слева-справа
            \end{columns}

          \column{.6\textwidth}
            Справа
        \end{columns}
      \end{verbatim}
  \end{columns}
\end{frame}
\begin{section}{Рекомендации по созданию презентаций}
  \begin{frame}{Для чего нужна презентация на защите}
    Задача презентации - пояснять устное выступление, а не отвлекать от него.
    \begin{itemize}
      \item Преимущественно нетекстовая информация.
      \item Краткий, чёткий, лаконичный текст.
      \item Дробление текста на списки.
    \end{itemize}
  \end{frame}
\end{section}

\end{document}

